\chapter{Conclusion}
In this chapter, I will conclude the results of my analysis and work. This includes a breakdown of what has been achieved the model-generating implementation, as well as suggestions to what can be achieved, by expanding upon the project in future work in Section \ref{sec:Future Work}. Section \ref{sec:Summary} contains a summary, describing key takeaways from the thesis in bullet-point form.
\section{Conclusions}
In Chapter \ref{chp:Background}, an assessment was provided of methods, which could be utilized to implement a solution to the problem that was also described in Chapter \ref{chp:Introduction}. Specifically, the methods, that were utilized in the solution were the combination of a parser and an interpreter, which would feed commands to a finite state machine, thereby creating the maritime models. This process of implementing this was described in Chapter \ref{chp:Work/Design}, and the process of determining this design is described in Chapter \ref{chp:Analysis}.

In Chapter \ref{chp:Results}, I verified the validity of the operations, performed in the implementation through extensive testing. That is, the testing, performed in Section \ref{sec:Testing}, concluded that the model creators, described throughout Chapter \ref{chp:Work/Design} could create working models, which adhered to the set of rules and descriptions, described in Chapter \ref{chp:Analysis}. 
\section{Future Work}

\subsection{Expandability}
%lazy evaluation
%modularity
%Monadic parser
\subsection{Automated Testing}
%Model-testing opposed to example-testing (Create a model, identify test-entities, do teststuff)

\section{Summary}

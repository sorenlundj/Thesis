\chapter{Introduction}
The Maritime Connectivity Platform~\cite{mcp} (MCP) is, as the name suggests, a platform that connects maritime services, functioning as a common infrastructure by offering safe and reliable information exchange between various maritime actors. A particular way of utilizing this, is that a user of the MCP is able to develop his or her own software, and distribute it across the world. Such digital maritime components, however, need to be thoroughly tested, which often is not done adequately through simple unit testing. Thus, software components, distributed through the MCP need a thorough test-suite, for which the obvious solution is to implement model-based testing.
The MCP has provided specifications, which describe rules that software needs to adhere to. These specifications will be used to create the model, that will verify the software. The specifications are provided in \ttt{xml}-format, and are, as of now, not used in any formal degree. Ideally, the provided specifications should be used to automatically generate models, which in turn, can be used to verify that a certain piece of software adheres to the specification. 
\section{Description of the Project}
In this project, model-based testing will be utilized, in order to verify that instances fulfills the requirements, set by specifications in the MCP. Model-based testing is a technique that allows for verification of overall \tit{model behavior}, rather than \tit{function correctness}. The idea behind this method is, as the term \tit{model}-based testing suggests, to create models which describe the properties that given test-cases need to reflect. 
Once an accurate model is created further testing can be streamlined, as a simple test can affirm an entire aspect of the tested software.
The goal of the project is to create both a manual and an automatic model-creating module for the MCP. The manual model creator will be run by the execution of commands, each creating different aspects and behaviors of the corresponding model, which, ultimately can be used for model-based testing. 
The automatic model-creator module, however, will draw information from the MCP specification xml-files, creating equivalent models, based on user-defined input. In the case of automatic model creation, the idea is that the originators of the maritime service will be the ones to author the xml-file, outlining the specifications, and in extension hereof create the maritime model.
\section{Learning Objectives}

\begin{itemize}
  \item Utilizing QuickCheck in creating viable models, that describe the specifications, provided in MCP specifications.
  \item Interpreting MCP specifications in order to generate relevant tests.
  \item Parsing MCP specifications in order to generate relevant models.
  \item Applying property-based testing of functions and specifications in a maritime environment.
\end{itemize}

\section{Scope}
This thesis will go into means of creating true and fair representations of models, representing maritime services. All tools, required to create models, both manually and automatically, will be analyzed, summarized and utilized, which will be described in the report.

\section{Limitations}
Methods of improvement upon both manual and automatic model-creation modules will be present, however not all will be utilized. This includes monadic parsers, and model verification through quickcheck, however in future work, these topics are relevant to be elaborated upon.

\section{Structure}
The remainder of the thesis will consist of~\ref{chp:Conclusion} chapters, covering the following topics:
\begin{description}
  \item[Chapter \refNumName{chp:Introduction}]\ \\
  This chapter provides a description of the problem statement, the learning objectives and the project itself, as well as containing this list.
  \item[Chapter \refNumName{chp:Background}]\ \\
  This chapter describes the techniques, theories, and components that are used or analyzed throughout the thesis.
  \item[Chapter \refNumName{chp:Analysis}]\ \\
  This chapter examines potential solutions to the problem statement, and how to implement them into the MCP. This chapter contains advantages and disadvantages to each presented solution, which are summed up at the end of the chapter.
  \item[Chapter \refNumName{chp:Work/Design}]\ \\
  This chapter presents the work that went into implementing the desired functionality in the MCP.
  \item[Chapter \refNumName{chp:Results}]\ \\
  This chapter includes the results that came from the implementation, described in the previous chapters.
  \item[Chapter \refNumName{chp:Discussion}]\ \\
  This chapter discusses the results, described in Chapter~\ref{chp:Results}.
  \item[Chapter \refNumName{chp:Conclusion}]\ \\
  This chapter concludes the thesis and compiles a short summary, highlighting central takeaways.
\end{description}
\chapter{Work/Design}

In this chapter I will describe the work I have done in order to implement model-based testing into the MCP. The workload that is contained in the implementation and description herein includes implementation of a monadic parser that reads the xml-specification files, an update of the fields and contents of fields in the xml-specification files, as well as a model, that is built upon the foundation of aforementioned updated xml-specification. %TODO Hvis det fungerer med interpreter(parser(xml))=model, så skriv det på.
\section{Updated Data}

As discussed in Section \ref{sec:Data}, in order to create uniform models, upon which to execute tests, the structure of the xml-specification files needs to be altered in a manner that adds uniformity to the xml-specification files. This rules out the use of natural language, for reliable model-generation.


%TODO: Evt en figur, der viser hvilke attributter, der indeholder hvilke atributter.

\section{Updated Parser Grammars}

\subsection{Updated Service Specification Grammar}

One way to implement these changes is make the user able to assign variables, based on a fixed collection of available options. An example of execution of this is given in Listing \ref{lst:sSpecUpd}. This parser grammar serves as an updated version of the parser grammar, described in Listing \ref{lst:sSpecFull}, however shaved for information\footnote{aSpec, authorInfo, authorInfos, contactInfo, dataExchangePattern, definitionAsXSD, description, isSpatialExclusive, keywords, oSpec, operation, operations, parameterType, parameterTypes, ptSpec, rSpec, requirement, requirements, returnValueType, serviceDataModel, serviceInterface, serviceInterfaces, siSpec, spec, specifications, status, text, typeReference, version}, that is irrelevant to the model-generating process. While this information can of course be included in the xml-specification file, it will be deemed irrelevant after parsing, and thus not included further in the model-generating process. The parser grammar presented in Listing \ref{lst:sSpecUpd} is designed to form a clear and unambiguous collection of entities, along with a similarly unambiguous description of relations, exchange-patterns, and security precautions. 

\begin{lstlisting}[keywordstyle={},label={lst:sSpecUpd},caption={Updated parser grammar of Service Specification Schema}]
ServiceSpecificationSchema ::= specifications

specifications ::= spec specifications
     | $\e$
     
spec ::= name
     | status
     | id
     | Entities

Entities ::= ESpec Entities
     | $\e$

ESpec ::= Ship
     | Service
     | Company

Ship ::= name
     | id
     | Relations

Service ::= name
     | id
     | Relations
     | Dependencies

Company ::= name
     | id
     | Relations

Relations ::= RSpec Relations
     | $\e$

RSpec ::= ESpec ESpec

Dependencies ::= DSpec Dependencies
     | $\e$

DSpec ::= Dependency RSpec
\end{lstlisting}

\subsection{Updated General Grammar}
The updated xml-specification files will be limiting the flexibility of the models in some aspects, as not all maritime services would follow the structure given in Listing \ref{lst:sSpecUpd}, however the example given is the earliest version of the domain-specific language. Following complex structures of various maritime services, the domain-specific language of the specification files can scale in complexity with the requirements set up by the maritime services.

Listing \ref{lst:sSpecUpdRed} describes a reduced form of Listing \ref{lst:sSpecUpd}. It can be seen that Listing \ref{lst:sSpecUpdRed} is identical to Listing \ref{lst:sSpecRed}, which proves that, after implementation of the model-generating process, there is no direct, urgent need to improve or change the parser.

\begin{lstlisting}[keywordstyle={}, label={lst:sSpecUpdRed}, caption={Reduced parser grammar of the updated Service Specification Schema}]
ServiceSpecificationSchema ::= specifications

specifications ::= spec specifications
     | $\e$
     
spec ::= specifications
     | spec
     | $\e$

spec ::= string
\end{lstlisting}

\section{Implementation in the MCP}

% Monadisk parsing

% Evt. lav en specifikationsXML selv.

% Evt lav nedenstående til et Chapter

\section{Technical Implementation}
\subsection{Executing Instructions}

\section{Testing}


